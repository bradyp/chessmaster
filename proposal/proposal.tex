\documentclass[a4paper,10pt,oneside,leqno,titlepage,onecolumn]{article}

\usepackage[utf8x]{inputenc}
\usepackage{graphicx}
\usepackage[parfill]{parskip}
\usepackage[left=1 in, right=1 in, top=1 in, bottom=1 in]{geometry}
\usepackage{hyperref}

\usepackage[framed,numbered,autolinebreaks,useliterate]{mcode}

\title{CSCE 489 - Introduction to Data Science \\ Final Project Proposal}
\author{Juan Burgos, Brady Pacha}
\date{\today}

\begin{document}
\maketitle

\section*{Project Goals}
Our goal is to design and implement a platform by which chess strategies can be analyzed 
in an in-depth and visual manner. We will provide an animated heat-map based using 
player turns as the "time" unit to show the probability distribution of specific chess 
pieces placement on the board and how it may change over time. Basically, we'll 
aggregate location data for each piece, over all of the games that we have on record, 
and will provide a "probability" heat-map of the piece's board location(s) at any given 
ingame turn. The user will be able to view changes in the heat-map's probability 
distribution using a slider.

Imagine a turn-by-turn animation that shows the probability of, for example a pawn, as 
the probability of it existing at any legal point on the board changes. If you want to 
know the location of some pawn after say 20 turns, simply pause the animation at the 
20th turn. This will show that, for example, the pawn has a 20\% probability of existing
at some point $(x,y)$ on the board, with the other 80\% distributed through some of 
other set of board locations. We will then provide that pawn's location probability 
distribution, visualized as a heat-map, over the entire board by analzying that pawn's 
location at the 20th turn for all 50,000 games.

\subsection*{Extra Goals - If we have ample time...}
Given a heat-map for any piece at any given turn, if we have time, we would like to try 
and provide survivability data for this piece, player win/loss predictions given any 
game state where the piece is at the current location given the current turn, average 
remaining game duration given the current board state(s), and/or a recommendation of the 
next best move.

Each of these items would be very interesting to investigate and possibly deliver to 
provide the user with additional information. Recall that these are items we'd like to 
explore \emph{if} we have enough time.


\section*{Project Motivation}
This application can be a useful educational tool for students learning to play chess. 
They can see what common piece placement strategies have been used in the past, 
including the most common locations of pieces after $x$ amount of turns and possibly 
some statistics associated with those piece placements. Chess enthusiasts may find an 
interest in the applicatino because it would provide a much deeper insight into 
strengths/weakness of strategies and relationships between different strategies.

\section*{Data}
\subsection{Data Format}
We have found several online repositories containing the list of all moves, including 
game meta-data, for roughly 50,000 chess games, many of which are master level matches. 
All of our data comes in PGN format, an example of which is provided below.
\lstinputlisting[language=Bash]{example.pgn}

\subsection{Example Source}
The following link is to one of the repositories that we visited: \href{http://
www.chess.com/download/view/world-chess-championship-1886-2012}{world-chess-
championship-1886-2012}. The package contains info on all games from the 1886 - 2012 
World Chess Championship tournaments in PGN format. 

\section{Current Design Ideas}
\subsection{Project Tools}
We will modify the source-code of the python parser called~\href{https://
pypi.python.org/pypi/pgnparser/1.0}{pgnparser} for extracting the useful information.
This will make up the "bigdata" portion of the project, possibly on AWS, where the 
system will distributively analyze the data from all 50,000 games and aggregate piece position histories into a useful format which we can actually use in our visualization.

\subsection{Implementation}
We've considered using D3 to produce the required visualizations for our application. A 
possibility that we are looking into is creating an interactive chessboard that will 
animate the probability distribution of any given piece after $x$ turns.

In the backend, we may use a database, possibly mongodb, to store the data required to 
produce an animated heat-map. 

The data needed to produce the heat-map will be processed via AWS by analyzing all the 
chess games to capture piece location statistics and histories.

\end{document}
